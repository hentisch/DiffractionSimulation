\documentclass[aspectratio=169]{beamer}
\usetheme{CambridgeUS}
\usepackage{graphicx}
\usepackage{animate}
\graphicspath{{./images/}{./plots/}}

\setbeamerfont{caption}{size=\tiny}
\setbeamertemplate{caption}[numbered]


\title {Simulating and Visualizing X-Ray Diffraction from Crystal Lattices}

\author{Henry Tischler}

\institute[Institute of Computing in Research]


\AtBeginSubsection[]
{
	\begin{frame}<beamer>{outline}
		\tableofcontents[currentsection, currentsubsection]
	\end{frame}
}
\begin{document}

\titlepage

\begin{frame}{X-Ray Crystallography}
\begin{columns}
				
	\begin{column}[b]{.5\textwidth}
		\begin{figure}
			\includegraphics[height=0.7\textheight]{photo_51}
			\caption{The famous Photo 51, a diffraction pattern of DNA which allowed it's structure to be deduced.}
		\end{figure}
	\end{column}
				
	\begin{column}[b]{.5\textwidth}
		\begin{figure}
			\includegraphics[height=0.7\textheight]{protein}
			\caption{A ribbon diagram of myoglobin, the structure of which was deduced via X-Ray crystallography.}
		\end{figure}
	\end{column}
				
\end{columns}
\end{frame}

\begin{frame}{Bragg's Law and Crystallography}
	X-Ray Crystallography relies on Bragg's Law, a property of the grids (lattices) crystall structures form.
	\begin{itemize}
	\item I am attempting to observe this property in a simulation of X-Ray scattering, without actually implementing it directly.
	\end{itemize}
	
	\begin{figure}
		\includegraphics[height=0.5\textheight]{lattice}
		\caption{A 3d visualization of a simple cubic crystal structure}
	\end{figure}
	
\end{frame}

\begin{frame}{What is Light?}
\begin{columns}
				
	\begin{column}[b]{.5\textwidth}
		When you have a changing electric field, a changing magnetic field will be created.
		\begin{itemize}
			\item A changing magnetic field will also create a changing electric field.
			\item This is why light is known as an electromagnetic wave.
		\end{itemize}		
	\end{column}
				
	\begin{column}{.5\textwidth}
		\begin{figure}
			\includegraphics[width=0.73\textwidth]{light}
			\caption{A visualization of a light wave}
		\end{figure}
	\end{column}
				
\end{columns}
\end{frame}

\begin{frame}{Thomson Scattering}
	Becuase light has an electric field, if it encounters an electron, it will wiggle the electron, causing the electron to scatter the light which encountered it.
	\begin{itemize}
		\item Note that this works with the electrons on an atom.
	\end{itemize}
	\begin{columns}
		\begin{column}{.5\textwidth}
			\begin{figure}
				\includegraphics[height=0.5\textheight]{amplitude}
				\caption {A visualization of Thomson Scattering from an electron.}
			\end{figure}
		\end{column}
		
		\begin{column}{.5\textwidth}
			\begin{figure}
				\includegraphics[height=0.5\textheight]{phase}
				\caption {A visualzation of Thomson scattering from an electron over space.}
			\end{figure}
		\end{column}
	\end{columns}
\end{frame}
		
\begin{frame}{Wave Interference}
	\begin{figure}
		\includegraphics[height= 0.7\textheight]{pond_wave_interference}
												
		\caption {Destructive Interference Occuring Between two Waves in a Pond}
	\end{figure}				
\end{frame}

\begin{frame}{Wave interference}
	When two waves meet, interference occurs.
	\begin{itemize}
		\item This interference is just the sum of the two waves.
	\end{itemize}
							
								
	\begin{columns}[c]	
		\begin{column}{.5\textwidth}
			\begin{figure}
				\includegraphics[height=0.55\textheight]{pi_shifted_matplotlib}
				\caption{Destructive Interference Between Two Sinesoids}
			\end{figure}
		\end{column}
											
										
											
		\begin{column}{.5\textwidth}
			\begin{figure}
				\includegraphics[height=0.55\textheight]{in_phase_matplotlib}
				\caption{Constructive Interference Between two Sinesoids}
			\end{figure}
		\end{column}
	\end{columns}
							
\end{frame}

\begin{frame}{Bragg's Law}
	Bragg's Law allows us to deduce the angle at which a beam of light needs to strike the surface of a crystall latice, in order to produce constructive interference.
	\begin{columns}
		\begin{column}{.3\textwidth}
			\begin{figure}
			$$ n \lambda = 2d sin \theta$$
			\caption{An equation modeling braggs law}
			\end{figure}
			
			\begin{tiny}
				\begin{itemize}
					\item $n$ can be any integer
					\item $\lambda$ is the wavelength of the incident light
					\item $\theta$ is the angle at which the incident light strikes the lattice 
				\end{itemize}
			\end{tiny}
		\end{column}
		
	\begin{column}{.7\textwidth}
	\begin{figure}
		\includegraphics[height=0.6\textheight]{braggs}
		\caption{A diagram of }
	\end{figure}
	\end{column}
	
	\end{columns}
\end{frame}

\begin{frame}{Simulating Diffraction Across a Lattice}
	To simulate this diffraction across a lattice, you simply compute the scattering for each atom at your observation point, and then find the sum.
	
	\begin{figure}
		\includegraphics[height=0.6\textheight]{lattice_simulation_points}
		\caption{The points used in the simulations of diffraction from a lattice}
	\end{figure}
	
\end{frame}

\begin{frame}{Successfull Results}
	Plots demonstrating Bragg's law were successfully created through the simulation	
	\begin{columns}
	\begin{column}{.5\textwidth}
	\begin{figure}
		\includegraphics[height=0.6\textheight]{constructive_atoms}
		\caption{A plot of an observation plane likely experiencing positive interference from braggs diffraction.}
	\end{figure}
	\end{column}
	
	\begin{column}{.5\textwidth}
		\begin{figure}
			\includegraphics[height=0.6\textheight]{nonconstructive_atoms_wavelength}
			\caption{A plot of an observation plane not experiencing positivie interference from Bragg's diffraction, as it's wavelength does not satisfy Bragg's law.}
		\end{figure}
	\end{column}
	\end{columns}
	
\end{frame}

\begin{frame}{Unsuccessfull Results}
	Certain plots did not successfully desmonstrate Bragg's Law, even though they theoerically should have.
	\begin{figure}
		\includegraphics[height=0.35\textwidth]{nonconstructive_viewpoint_4cm}
	\end{figure}
\end{frame}

\begin{frame}{Insights}
	The main takeaway from this simulation is that Bragg's law is simply the summation of Thomson scattering.
	\begin{itemize}
		\item This is shown by observing Bragg's Law, simply by implementing Thomson scattering from a crystal lattice.
	\end{itemize}
	
	Even though Bragg's Law could be shown more perfectly this type of simulation, it is certainly not easy.
		\begin{itemize}
			\item To get a truly accurate view of Bragg's Law simply from a simulation of scattering, the simulation would have to have very little error.
		\end{itemize}
\end{frame}

\begin{frame}[allowframebreaks]
	\frametitle<presentation>{Bibliography}
	
	\begin{thebibliography}{10}
	
	\beamertemplatebookbibitems
	
	\bibitem{*}
		\newblock Elements of Modern X-Ray Physics
		\newblock {\em Jens Als-Nielsen, 2001}
		
	\bibitem{*}
		\newblock X-Ray Diffraction
		\newblock {\em B.E Warren, 1990}
	
	\end{thebibliography}
\end{frame}

\end{document}