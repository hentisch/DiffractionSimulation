\documentclass[aspectratio=169]{beamer}
\usetheme{CambridgeUS}
\usepackage{graphicx}
\graphicspath{{./images/}{./plots/}}

\setbeamerfont{caption}{size=\scriptsize}

\title {Simulating and Visualizing X-Ray Diffraction from Crystall Lattices`}

\author{Henry Tischler}

\institute[Institute of Computing in Research]


\AtBeginSubsection[]
{
	\begin{frame}
		\tableofcontents[currentsection, currentsubsection]
	\end{frame}
}

\begin{document}
	\titlepage


\begin {frame}{What is Light?}
	When you have a changing electric field, a changing magnetic field will be created.

	\begin{itemize}
	\item A changing magnetic field will also create a changing electric field.
	\item These changing electric fields will form the wave that is light.
	\end{itemize}		
\end{frame}

\begin {frame}{What is Light?}
	%\includegraphics[width=\textwidth]{/home/henry/DiffractionSimulation/materials/slideshow/images/light.png}
\end{frame}


\begingroup
\begin{frame}{Wave Interference}
	\begin{figure}
		\includegraphics[height= 0.7\textheight]{pond_wave_interference}

	\caption {Destructive Interference Occuring Between two Waves in a Pond}
	\end{figure}

\end{frame}
\endgroup

\begin{frame}{Wave interference}
	When two waves meet, interference occurs.
	\begin{itemize}
	\item This interference is just the sum of the two waves.
	\item This means that when the waves are in phase with eachother, their valleys and peaks will line up, and the waves will constructively interfere.
	\item When the waves are perfectly out of phase, the valleys of one wave will line up with the peaks of another, and the waves will cancel eachother out.
	\end{itemize}
	

\begin{columns}[c]	

	\begin{column}{.5\textwidth}
		\begin{figure}
		\includegraphics[height=0.2\textheight]{pi_shifted_matplotlib}
		\caption{Destructive Interference Between Two Sinesoids}
		\end{figure}
	\end{column}
	

	
	\begin{column}{.5\textwidth}
		\begin{figure}
			\includegraphics[height=0.2\textheight]{in_phase_matplotlib}
			\caption{Constructive Interference Between two Sinesoids}
		\end{figure}
	\end{column}
\end{columns}

\end{frame}

\begin{frame}{Refrences}
	"Two point source interference Pattern" by Lub0t is licensed under https://creativecommons.org/licenses/by/3.0/
\end{frame}

\end{document}